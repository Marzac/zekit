% !TEX TS-program = pdflatex
% !TEX encoding = UTF-8 Unicode

%%% DOCUMENT DEFINITIONS
% \documentclass[11pt]{article}	% use larger type; default would be 10pt
\documentclass{scrartcl}
\setkomafont{author}{\scshape}
\usepackage{blindtext}
\usepackage[utf8]{inputenc}	% set input encoding (not needed with XeLaTeX)

%%% PAGE DIMENSIONS
\usepackage{geometry}		% to change the page dimensions
\geometry{a4paper}		% or letterpaper (US) or a5paper or....
\geometry{margin=3cm}		% for example, change the margins to 2 inches all round

%%% PACKAGES
\usepackage{wrapfig}
\usepackage{textcomp}
\usepackage{gensymb}
\usepackage{graphicx} 		% support the \includegraphics command and options
\usepackage{booktabs} 		% for much better looking tables
\usepackage{array} 		% for better arrays (eg matrices) in maths
\usepackage{url}
\usepackage{enumitem}
\usepackage{longtable}
\usepackage[defaultlines=4,all]{nowidow}
\usepackage{multicol}

\usepackage{tikz}
\newcommand*\circled[1]{\tikz[baseline=(char.base)]{
        \node[shape=circle,draw,inner sep=2pt] (char) {#1};}}

% Make TOC and URLs clickable
\usepackage[
    colorlinks,
    pdfborder={0 0 0},
    linkcolor=black,
    citecolor=black,
    filecolor=black,
    urlcolor=blue
]{hyperref}

%%% Adjust paragraph indent and spacing
\usepackage{parskip}

%%% HEADERS & FOOTERS
\usepackage{fancyhdr} 		% This should be set AFTER setting up the page geometry
\pagestyle{fancy} 			% options: empty, plain, fancy

%%% More compact item lists
\setlist[itemize]{itemsep=-3pt,topsep=0pt}

%%% TITLE PAGE
\title{
    \vspace*{4cm}
    \huge{zekit} \\
    Instruction and User Guide \\
    \vspace*{0.25cm}
    \small{Revision 1.0 EN - 17/05/2021} \\
    \small{for Firmware V1.00 - 17/05/2021} \\
    \vspace*{0.5cm}
    (Todo: cover image)
    % \includegraphics[scale=0.9]{assets/panel.png}
}
\author{Frédéric Meslin / Fred's Lab}

%%% DOCUMENT
\begin{document}

\maketitle

\pagebreak

% ------------------------------------------------------------------------------------------

%%% TABLE OF CONTENTS
\tableofcontents
\pagebreak

% ------------------------------------------------------------------------------------------

%%% INTRODUCTION
\section{Introduction}

\textbf{Thank you very much for purchasing the zekit!}

The \textbf{zekit} is a 4-voice paraphonic synthesizer kit with digital sound synthesis and an analog filter and VCA. It also features two fully analog envelopes and can be controlled via MIDI and additional clock signals.

You'll learn a lot while building this kit and you'll end up with a cheap and simple but fun instrument that will integrate into your studio setup.

\subsection{Required Skills}

\begin{itemize}
    \item Basic knowledge of analog synthesis
    \item Soldering experience with through-hole (THT) components
    \item Knowledge how to interpret a placement diagram
    \item Knowledge of electronics at an intermediate level
\end{itemize}

\subsection{Required Tools}

In order to fully assemble the kit, you'll need the following tools:

\begin{itemize}
    \item A soldering iron with a tip size around 3mm
    \item Solder with a diameter around 1mm
    \item A wire cutter
    \item A screwdriver
    \item A multimeter (todo: check if required)
\end{itemize}

\subsection{Required Accessories}

\begin{itemize}
    \item Power supply DC 5-9V, 0.5A with barrel connector 5.5mm x 2.5mm, center positive
    \item MIDI cable
    \item Audio cable
    \item Computer or MIDI controller
    \item Audio system
\end{itemize}

% ------------------------------------------------------------------------------------------

\section{Legal notices}

\textbf{Fred's Lab} cannot be liable for erroneous information contained in this manual. The contents of this manual may be updated at any time without prior notice. We have put best effort to ensure the information provided here is useful and accurate. Fred's Lab extends no liabilities in regard to this manual other than those required by the local laws.

\begin{center}
    \textbf{Frédéric Meslin Audiogeräte} \\
    Herwarthstraße, 20 \\
    53115 Bonn, Germany \\
    \url{info@fredslab.net} \\
    \url{http://fredslab.net} \\
\end{center}

\textbf{Support requests} \\
For support requests, you can reach me per e-mail at:
\begin{center}
    \url{support@fredslab.net}
\end{center}
or per post, using the previously mentioned company address.

For each support request, please include the instrument model, serial number and a precise description of the problem encountered with a maximum of details and supporting elements for a quick resolution.

\textbf{Copyright information}

This original manual, its content, including the graphics \& descriptions are the property of \textbf{Fred's Lab}. No part of this manual should be reproduced other than for customer personal use and backup needs without a written permission from \textbf{Fred's Lab}.

\pagebreak

% ------------------------------------------------------------------------------------------

\section{Warranty}
\textbf{Fred's Lab} warranty this product free of defects \textbf{3 years} from its
date of purchase.

This warranty covers product from manufacturing defects, when the product is used observing normal operating conditions. However, the warranty \textbf{does not cover}:

\begin{itemize}
    \item Normal product wear-out
    \item Damages caused by failure to observe the rules of use
    \item Damages due to negligence of the user
    \item Products having been modified or repaired by the user or a third person
\end{itemize}

More information about product warranty can be found in the \textbf{General Terms and Conditions of Sale document} available at:
\begin{center}
    \url{https://fredslab.net/en/terms.html}
\end{center}

% ------------------------------------------------------------------------------------------

\section{Special Thanks}

I would like to thank the following persons for their contributions to this project:

\begin{itemize}
    \item Quality insurance: Benoit Ruelle, René Schmidt, Mathieu Meslin, Oliver Rockstedt
    \item Instruction guide: Oliver Rockstedt
\end{itemize}

% ------------------------------------------------------------------------------------------

\section{Precautions}
Before plugging in the \textbf{zekit!} and go \textbf{rocking the world}, have a sit and read this precautions through:

\begin{itemize}
    \item Always use the device in a dry and warm environment
    \item Never drop the device or expose to too much pressure or vibration
    \item Never spill liquids or bath the device in beer
    \item Never clean the device with an aggressive solvent
    \item Never wiggle the plugs to disconnect the cords
    \item Never connect the line outputs to the power outputs of an amplifier
    \item Only modify the unit at your own risk!
\end{itemize}

\textbf{The zekit} used in conjunction with headphones and speaker systems can produce \textbf{very loud sounds} in a wide range of frequencies.

Human hearing is \textbf{very sensitive} and can be damaged quickly. So watch out your hears and those of your audience!

\pagebreak

% ------------------------------------------------------------------------------------------

\section{Box Content}

\begin{itemize}
    \item PCB with SMD components pre-assembled
    \item 7 tactile switches
    \item 2 toggle switches
    \item 2 potentiometers 10k
    \item 4 potentiometers 100k
    \item 1 power switch
    \item 1 power jack
    \item 1 MIDI jack
    \item 1 audio output jack (6.3mm)
    \item 2 jacks (3.5mm) for clock and audio input signals
    \item 3 film capacitors 47nF
    \item 6 potentiometer knobs
    \item PIC microcontroller IC
    \item Optocoupler IC for VCF/VCA
    \item Optocoupler IC for MIDI
    \item 3 sockets for ICs
    \item Enclosure plates
\end{itemize}

\subsection{The Pre-Assembled PCB}

\subsection{Provided Parts}

\subsubsection{Tactile Switches}

\begin{center}
    \includegraphics[scale=0.5]{assets/zekit-tacts-resized.jpg}
\end{center}

\subsubsection{Toggle Switches}

\begin{center}
    (Todo: image)
\end{center}

\subsubsection{Potentiometers}

\begin{center}
    \includegraphics[scale=0.5]{assets/zekit-pots1-resized.jpg}
    \includegraphics[scale=0.5]{assets/zekit-pots2-resized.jpg}
\end{center}

\subsubsection{Power Switch}

\begin{center}
    \includegraphics[scale=0.5]{assets/zekit-powerswitch-resized.jpg}
\end{center}

\subsubsection{Power Jack}

\begin{center}
    \includegraphics[scale=0.5]{assets/zekit-dcjack-resized.jpg}
\end{center}

\subsubsection{DIN Jack}

\begin{center}
    \includegraphics[scale=0.5]{assets/zekit-din-resized.jpg}
\end{center}

\subsubsection{6.3mm TRS Jack}

\begin{center}
    \includegraphics[scale=0.5]{assets/zekit-jack1-resized.jpg}
\end{center}

\subsubsection{3.5mm TRS Jacks}

\begin{center}
    \includegraphics[scale=0.5]{assets/zekit-jack2-resized.jpg}
\end{center}

\subsubsection{Film Capacitors}

\begin{center}
    \includegraphics[scale=0.5]{assets/zekit-filmcaps-resized.jpg}
\end{center}

\subsubsection{Potentiometer Knobs}

\begin{center}
    \includegraphics[scale=0.5]{assets/zekit-knobs-resized.jpg}
\end{center}

\subsubsection{Microcontroller IC}

\begin{center}
    \includegraphics[scale=0.5]{assets/zekit-mcu-resized.jpg}
\end{center}

\subsubsection{Optocoupler ICs}

\begin{center}
    (Todo: images)
\end{center}

\subsubsection{IC Sockets}

\begin{center}
    (Todo: correct image)
    \includegraphics[scale=0.5]{assets/zekit-sockets-resized.jpg}
\end{center}

\subsubsection{Enclosure Plates}

\begin{center}
    (Todo: image)
\end{center}


\pagebreak

% ------------------------------------------------------------------------------------------

\section{Assembling the Kit}

\subsection{General Advices}

\subsection{Setting Up the Soldering Gig}

\subsection{Soldering the Components}

Order of soldering:

\begin{itemize}
    \item IC sockets
    \item Film capacitors
    \item Trimmers
    \item Tactile switches
    \item Toggle switches
    \item 10k Potentiometers
    \item 100k Potentiometers
    \item Connectors
\end{itemize}

Important remarks:

\begin{itemize}
    \item Polarity of switches
    \item Polarity of sockets (matching the IC direction)
    \item- Taping for easy soldering
    \item Not heat film caps for too long
    \item Solder pots and switches STRAIGHT
          (first solder one leg, press it, then the rest)
\end{itemize}

\subsection{Installing the ICs}

(Todo: maybe power up and measure supply voltages before)

\begin{itemize}
    \item Operational Amplifier
    \item Optocoupler
    \item Microcontroller
\end{itemize}

\section{Power Up and Test}

 (Todo: what to expect)

\subsection{Calibration of the VCF}

(Todo: adjusting the trimmers)

\subsection{Identifying Common Issues}

\begin{itemize}
    \item Bad soldering
    \item Wrong polarity (Todo: of what? ICs?)
\end{itemize}

\pagebreak

% ------------------------------------------------------------------------------------------

\section{Functional Description}

\subsection{Power Supply}

The Zekit needs the energy delivered by the power supply. This block provides 2 voltages: +3.3V and -2V, using a linear regulator and a charge pump respectively.

\subsubsection{Linear Regulator}

\begin{center}
    \includegraphics[scale=0.4]{assets/schema-power.png}
\end{center}

\emph{SW1} is the power switch. \emph{D3} ensure voltage polarity is correct, by letting the current flows only in the right direction. \emph{C4}, \emph{R9} \& \emph{C7} filter the input voltage, reducing supply noise. \emph{U2} regulates down the input, assuring a stable +3V3 in all circumstances. \emph{C8} secures \emph{U2} against possible oscillation of its output voltage.

\subsubsection{Charge Pump}

\begin{center}
    \includegraphics[scale=0.3]{assets/schema-pump.png}
\end{center}

The \emph{PUMP} signal is fed by the microcontroller and alternates between GND and +3.3V at a high frequency. When \emph{PUMP} is at +3.3V, \emph{C3} is charged via \emph{R3} and \emph{D2}, while \emph{D4} is non-conductive. When \emph{PUMP} is at GND, \emph{D2} is isolating and the charge of \emph{C3} is transferred to \emph{C5} via the now conducting \emph{D4}. The inductor \emph{L1} filters the voltage for better noise immunity.

Ideally, the output voltage at \emph{TP1} would be -3.3V, but because of the loss inside the diodes, it will be around -2V, which is still sufficient for proper operation.

\subsection{Microcontroller}

\begin{center}
    \includegraphics[scale=0.55]{assets/schema-mcu.png}
\end{center}

The microcontroller (MCU) \emph{U7} is the brain of the \textbf{zekit}. It processes all signals from the switches and the inputs and generates the audio waveform as well as some control signals for the analog circuitry. This is done by running a firmware that is stored in the internal flash of the chip. The firmware is pre-programmed by Fred's Lab, but can be replaced by using a dedicated programmer kit.

The capacitors \emph{C11}, \emph{C12} and \emph{C14} are used together with the inductor \emph{L2} to improve the stability of the power supply and reduce noise.

\subsection{DAC}

\begin{center}
    (Todo: DAC schematics)
    % \includegraphics[scale=0.55]{assets/schema-dac.png}
\end{center}

The digital-to-analog converter (DAC) \emph{U3} is directly connected to the microcontroller via 3 signals. It converts the waveform calculated by the MCU into the analog domain. Capacitor \emph{C13} removes the DC offset of the signal, resistor \emph{R19} conditions the level to fit in the range of the next stage (the VCF).

The DAC chip provides two independent output channels of which only one is used internally. The unused channel is routed to the test point \emph{TP4} and can be utilized for additional purposes.

\subsection{VCF}

\begin{center}
    \includegraphics[scale=0.39]{assets/schema-vcf.png}
\end{center}

The filter is a 2nd order design with a 12dB/octave slope. The first stage consists of the optocoupler \emph{U3a/U3b} and the capacitor \emph{C15}, the second one of optocoupler \emph{U3b} and capacitor \emph{C19}, in which the optocouplers take the role of the resistive element to control the frequency.

The input signal \emph{VCF-IN} is decoupled by \emph{C13} and fed in either to the first stage or the feedback path, depending whether low-pass or band-pass configuration is selected by toggle switch \emph{TG1}. The resonance is determined by the feedback loop and can be chosen between two amounts via toggle switch \emph{TG2}. The trimmer \emph{TR2} allows to fine-control one of the settings. The operational amplifier \emph{U6b} provides the necessary gain for driving the next stage and the feedback loop. It utilizes the diode \emph{D7} to tame the amplitude at higher levels by producing a nice-sounding distortion.

\subsection{VCA}

\begin{center}
    \includegraphics[scale=0.37]{assets/schema-vca.png}
\end{center}

The VCA uses the optocoupler \emph{U3c/U3d} to gain-control the operational amplifier \emph{U6a}. The input signal \emph{VCA-IN} is decoupled by capacitor \emph{C20} and then divided down by resistors \emph{R39} and \emph{R40} to an appropriate level for the optocoupler. To set the output level, the \textbf{Volume} potentiometer is located in the feedback path of the opamp. Resistor \emph{R47} in parallel to the pot provides a nicer control curve for the volume setting. The output is filtered by inductor \emph{L1}, followed by protection resistors \emph{R51} and \emph{R52} and then routed to the output jack \emph{J6}.

\subsection{Envelopes}

\subsubsection{VCF Envelope}

\begin{center}
    \includegraphics[scale=0.36]{assets/schema-ar.png}
\end{center}

The filter envelope is generated by either charging capacitor \emph{C23} to +3.3V or discharging it to GND. To start the envelope, the microcontroller sets the \emph{GATE-VCF} signal to +3.3V. The transistor \emph{Q5a} then conducts and \emph{C23} is charged via the resistor \emph{R42} and the \textbf{Attack} potentiometer. To put the envelope in release, the MCU sets the \emph{GATE-VCF} signal to GND. \emph{Q5a} then cuts off and \emph{Q5b} starts to conduct, resulting in \emph{C23} being discharged via \emph{R43} and the \textbf{Release} potentiometer.

The transistor \emph{Q7} acts as a buffer to decouple the envelope. The \emph{ENV-VCF} signal is routed back to the MCU so it can detect that the attack phase is finished. Finally, the envelope amount is set by the \textbf{Accent} potentiometer and added to the constant level set by the \textbf{Cutoff} potentiometer.

\subsubsection{VCA Envelope}

The amplifier envelope works in the same ways as the VCF envelope, but lacks the attack control.

\subsection{Exponential Control Circuit}

\begin{center}
    \includegraphics[scale=0.42]{assets/schema-expo.png}
\end{center}

This circuit is used to convert the linear control voltage into an exponential current for the optocouplers. The requirement for this is the nature of our human hearing which recognizes pitches and volumes in an exponential way.

The input of this block uses the signal \emph{VCF} from the filter envelope to drive the buffer transistor \emph{Q1a}. The trim potentiometer \emph{TR1} is used to set a bias level for calibration which is added to the input. The emitter of \emph{Q1a} drives the base of transistor \emph{Q1b} which does the curve conversion. It is based on the fact that the relationship between the base-emitter voltage and the collector current of a transistor is exponential. The collector current of \emph{Q1b} drives the LEDs inside the optocoupler \emph{U5} which controls the filter cutoff.

The VCA uses a similar circuit that is feeded by the VCA envelope.

\subsection{MIDI Input}

\begin{center}
    \includegraphics[scale=0.40]{assets/schema-midi.png}
\end{center}

The MIDI input is electrically isolated with an optocoupler to prevent ground loops. This is required by the official MIDI specification. The external device drives the LED inside the optocoupler \emph{U1} via the resistor \emph{R5}. When the LED is on, the photo transistor inside the optocoupler conducts and the \emph{MIDI-IN} signal is pulled to GND. When the LED is off, the photo transistor cuts off and \emph{MIDI-IN} is set to +3.3V by the pullup resistor \emph{R10}.

\subsection{Clock Input}

\begin{center}
    \includegraphics[scale=0.40]{assets/schema-clocks.png}
\end{center}

The clock signal is taken from the tip of the input jack \emph{J4}. If no plug is inserted, a high level is preset by the pullup resistor \emph{R57}. Depending on the input voltage, transistor \emph{Q4} is either conducting or not. Resistor \emph{R12} and diode \emph{D6} protect the transistor against over-voltage or wrong polarity at the input. At the end, the signal \emph{CLK-IN} is then routed to the microcontroller. The MCU uses an internal pullup activated by the firmware to detect the high level when transistor \emph{Q4} is off.

The clock start signal is taken from the ring of the input jack and processed in a similar way by driving the transistor \emph{Q3} to generate the \emph{CLK-START} signal for the MCU.

\subsection{Tactile Switches}

\begin{center}
    \includegraphics[scale=0.25]{assets/schema-switch.png}
\end{center}

The 7 illuminated tactile switches are directly connected to the microcontroller. The image above shows the \emph{WAVE} button switch \emph{SW2} as an example.

In order to reduce the number of required signal lines, the MCU firmware uses a neat trick: it alternates the mode of the pin connected to the \emph{BUT-WAVE} signal between input and output.

When the MCU pin is configured as input, the button state can be read. In case the button is released, a high level is detected because the pin is connected to +3.3V via the LED and the resistor \emph{R18}. When the button is pressed, the pin is directly connected to GND and a low level is detected. When the MCU pin is configured as output, the LED can be illuminated by driving the signal to GND.

\pagebreak

% ------------------------------------------------------------------------------------------

\section{Additional Resources}

=> The github
=> The webpage
=> Where to find the MCU source code

\pagebreak

% ------------------------------------------------------------------------------------------

\section{Legal / License}

\pagebreak

% ------------------------------------------------------------------------------------------

\section{Operation}

 (Todo: user manual)

% ------------------------------------------------------------------------------------------

\section{Schematics}

To minimize electronic waste and ensure long product life, Fred’s Lab is willing to provide all technical documents needed to repair his products. The following schematics are provided ”as is” with no warranty of any kind. Any modification made to a Fred’s Lab instrument immediately voids the included 3-year product warranty. Repairs must be carried out by a competent repair service. Fred’s Lab stays available for the maintenance of your instruments. Do not hesitate to contact the support service for a free quote. Spare parts can directly be ordered from us.

\textbf{Intellectual property:}

The following technical documents are provided for advisory, repair and educational purposes only. They remain the entire property of Fred's Lab and cannot be reproduced without a written authorization. Users are granted to draw inspiration from this information for their projects (commercial or not), while respecting the limits of non-cloning or counterfeiting the original product. If in doubt about legal matters, please contact us.

\begin{center}
    (Todo: full schematics)
    % \includegraphics[scale=0.7,angle=90,origin=c]{assets/schema-full.png}
\end{center}


\end{document}
